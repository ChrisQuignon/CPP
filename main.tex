\documentclass[a4paper]{article}
\usepackage[latin1]{inputenc}
\usepackage[UKenglish]{babel}
\usepackage[T1]{fontenc}
\usepackage{lmodern}
\usepackage[english=american]{csquotes}
\usepackage{url}
\usepackage{latexsym}
\usepackage{graphicx}
\usepackage{endnotes}
\usepackage{amsmath}
\usepackage{amssymb}
%\usepackage{math}
\usepackage{float}
\usepackage{longtable}

%\usepackage[breaklinks]{hyperref}

%Graphic
%including macros only
%\let\footnote=\endnote

\newcommand{\TODO}       [1]{\textbf{\emph{[TODO: #1]}}}

%Graphic
\newcommand{\graphic}[3]{
\begin{figure}[H]
\begin{center}
   \includegraphics[width=#2cm]{graphics/#1.png}
   \label{fig:#1}
   \caption{#3}
\end{center}
\end{figure}
}

\newcommand{\sqt}{$\sqcap$}
\newcommand{\sqb}{$\sqcup$}
\newcommand{\sqr}{$\sqsupset$}
\newcommand{\sql}{$\sqsubset$}

\renewcommand{\tt}{$\vartriangle$}
\newcommand{\tb}{$\triangledown$}
\newcommand{\tr}{$\vartriangleright$}
\newcommand{\tl}{$\vartriangleleft$}

\newcommand{\boxit}[3]{
\\
\begin{figure} [H]
	\begin{center}
		\fbox{\parbox{12.5cm}{
		\par
		\begingroup
		\leftskip=0.5cm
		%\noindent
		#1
		\endgroup}}
	\label{ref:#2}
	\caption{#3}
	\end{center}
\end{figure}
}

%List
%\begin{itemize}%enumerate
%	\item Punkt1
%	\item Punkt2
%\end{itemize}%enumerate


%Ref to Picture:
%(see:\ref{fig:XXX})

%Table
%\begin{table}[H]
%\begin{longtable} {| l l  | }
%\hline
%Topic1		& Topic2 	\\ \hline
%Cell1		& Cell2		\\
%Cell1.1	& Cell2.1	\\
%\hline
%\end{longtable}
%\end{table}


\begin{document}
\begin{titlepage}


\title{%Cell Processor Programming\\
	\Huge Software for the Cell processor\bigskip\\ \Large Patterns in aligned prime numbers\\}
\author{Christophe Quignon}
\date{Otto-Friedrich University of Bamberg\\
Software Technologies Research Group\\
\vskip 1 cm	
\today}

\maketitle
\vskip 6.5 cm
\begin{abstract}
This report depicts the design and implementation process of a software that exploits the Cell processor architecture. The theme of the software is the alignment of prime numbers in a two dimensional space, similar to the Ulam spiral. To locate and describe the difference between parallel and serial software, both versions were designed, but only the serial version was implemented successfully.
\end{abstract}
\thispagestyle{empty}
\newpage
\tableofcontents
\pagenumbering{Roman}
\newpage
\listoffigures
\newpage
\end{titlepage}
\pagenumbering{arabic}
\section{Introduction}
\label{sec:intro}

The Electronic Frontier Foundation awarded \$100,000 for the first discovered prime number with at least 10,000,000 digits and the Great Internet Mersenne Prime Search ("GIMPS") won this prize in October 2009. \$100,000 and \$250,000 are awarded to the discovery of the first prime numbers with 100,000,000 and 1,000,000,000 decimal digits.

$2^{43,112,609}-1$ is the largest currently known prime number. Its discovery consumed 12 Years of continuous computing time with 29 trillion calculations per second and lead to 12,978,189 digits in decimal notation and was discovered by the Great Internet Mersenne Prime Search ("GIMPS").
Computing with prime numbers in general isn't trivial. Their size easily exceeds normal integer limitations and the algorithms to find and prove them are hard.

The reason behind this monetary appeal is the unsolved question if prime numbers follow a specific pattern to which they appear. This is one of the great question of number theory and up to a few years ago it was so solely. But since the integer factorisation is the core of every encryption, is a question of great interest for computer science.  

Euclid has proven that there are infinitely many prime numbers, but there is no function which creates all possible prime numbers. Maybe this function can not exist, maybe this question is not decideable.\cite{zahlentheorie} However, it is possible to tell where between two prime numbers a gap of a given size is. These prime gaps prompt other mathematical questions but may give a hint to a pattern.

Another big discovery towards a pattern of prime numbers is the Ulam Spiral. In 1963 Stanislav Ulam was bored and started marking prime numbers on a square spiral as in \ref{fig:ulam_spiral_numbers}. \ref{fig:ulam_400_400} show a 400 by 400 Ulam spiral. At angles of 45° lines evolved. Up to now, it is not clear why these lines appear, but many people started their own search and came up with variation of the Ulam Spiral. The starting number may vary, the process was applied iterativel\cite{web} and other ways of illustration were chosen\cite{web}.


\begin{figure}[H]
\begin{minipage}[t]{0.475\textwidth}
\centering
    \graphic{ulam_spiral_numbers}{5}{The Ulam Spiral center.} 
\end{minipage}
\hfill
\begin{minipage}[t]{0.475\textwidth}
\centering
   \graphic{ulam_400_400}{5}{A 400x400 Ulam Spiral.}
\end{minipage}
\end{figure}

The search for patterns, the primality test, the vast search space where these patterns may evolve and the necessitiy to verify once found regularities against various variables is a fine problem to solve in parallel.
 
%\subsection{subsection}
%\subsubsection{subsubsection}
%\paragraph{paragraph}
%\subparagraph{subparagraph}


%\subsection{subsection}
%\subsubsection{subsubsection}
%\paragraph{paragraph}
%\subparagraph{subparagraph}
\section{Design}
\label{sec:design}
The design of the software was aligned to the given requirements and will be presented in the same order. In a conflating step, the given design deliberations are merged to an overall design.



\subsubsection{Primality tests}
\label{sec:primality_test}
A more theoretical problem is the primality test. It tests if a given number \emph{n} is a prime number, but does not lead to its prime factors and therefor is a bit faster than integer factorization. This tests applied to a list of all integers leads to a list of all primes and is therefore suitable to create an Ulam spiral. Probabilist test hold a chance for false positives, for their prove is restricted with a probability. An error within this probability may lead to unintended patterns in the Ulam spiral. This problem is considered in section \ref{sec:prim_test} where the implementation of the primality test is discussed.

Before the decision, which primality test is used, some of these test were examined:

\paragraph{Naive primality test}
The simplest and less effective primality test for the number $n$ is to check all numbers up to \emph{n}, if \emph{n} is divisible by any of these numbers. A small improvement would be to  test just numbers $< \sqrt{n}$, because at least one of the factors of \emph{n} hast to be in this range.

%A famous deterministic primality test is the "Adleman-Pomerance-Rumely primality test". It was later improved by Henri Cohen and Arjen Lenstra and can test primality of an integer n in time:
    $(\log n)^{O(\log\,\log \,\log n)}$. 

%\paragraph{The AKS primality test}, also known as Agrawal-Kayal-Saxena primality test ord cyclotomic AKS test is a deterministic primality-proving algorithm and was published in \cite{primeisp}. And is descibed as follows:

%\begin{quote}
%Input:\\
%\noindent\hspace*{12mm} integer n $>$ 1.\\
%1. If (n = ab for a $\in$ N and b $>$ 1), output COMPOSITE.\\
%2. Find the smallest r such that or (n) $>$ log2 n.\\
%3. If 1 $<$ (a, n) $<$ n for some a $\leq$ r, output COMPOSITE.\\
%4. If n $\leq$ r, output PRIME.1\\
%5. For a = 1 to $\phi$(r) log n do\\
%   \noindent\hspace*{12mm} if ((X + a)n = X n + a (mod X r - 1, n)),\\
%   \noindent\hspace*{24mm}output COMPOSITE;\\
%6. Output PRIME;
%\end{quote}

\paragraph{Sieve of Eratosthenes}
An alternative way of generating all prime numbers is the so called "Sieve of Eratosthenes". It works efficient for smaller primes.

From a list of all natural numbers $> 2$ the three following steps will reveal all prime numbers:

\begin{itemize}%enumerate
   \item The first number in the list is a prime
   \item Strike this number and all multiples of the current number from the list
   \item repeat  
\end{itemize}%enumerate

This test scales quite well for small numbers up to 10,000. The test of every number includes the testing of all numbers below which results in massive computation overhead if the number exceeds the algorithms scalability.

\paragraph{Miller-Rabin probabilistic primality test}
The Miller-Rabin probabilistic primality test was first published in \cite{MRPT} and verified in \cite{MRVer}. The running time of this algorithm is $O(k log^{3} n)$, where k is the number of different values tested. Therefore it is quite performant while have the chance of false positives.

\subsubsection{Number size}
\label{sec:number_size}
As proven by Euclid, there are infinitive many prime numbers. To verify possible patterns, the large prime numbers are of special interest. Only if a regularity also holds with large numbers, it is of interest to proof this regularity mathematically.
The prime number theorem (PNT) describes the asymptotic distribution of the prime numbers. The PNT, based on the formula of prime distribution by Gauss, gives an approximation of the density of prime numbers. According to Chebyshev, this density seems do decrease with larger numbers, as there are larger and larger prime gaps. This may confirm regularities and also may be conformed by found regularities.

From the viewpoint of a programming language, large numbers include growing complexity in their handling. In theory, every number is only restricted by the memory size that holds the number. In practice, the limiting factor is the standard which defines the representation of the number and the size of the smallest cache the number passes. This normally is limited to 32bit on old processor architectures and 64 bit in modern architectures like the PS3.

For prime numbers, the problem only scales on the set of all positive natural numbers, which is quite smaller problem then simulating continuous numbers and handling discretization errors. The  very common problem of large natural number handling is solved by various libraries, but has to be implemented with care. Buffer overflows on too small number ranges highly harm the result of the computation and have to be avoided.

\subsubsection{Testing}
\label{sec:tests}
To have a basis on which the pattern searches may be performed, several changes to the original Ulam spiral were taken in account:

\begin{itemize}%enumerate
   \item Starting number
   \item Size of the space
   \item Curve
   \item Number sequence
   \item Dimension of the curve   
\end{itemize}%enumerate

In rising complexity and time consumption, the last two point were dropped to future works, but the starting number, the size of the space and another two curves were implemented and tested.

\subsubsection{Storage}
\label{sec:concurrency}
The architecture of the PS3 with its Synergistic Processing Elements (SPE) limit the amount of data that can be processed. 256KB embedded SRAM have to store the instructions and data.$^{\cite{PTC}}$
Large data has to be broken down to chunks of at most this size. Context switches and data reloads will slow down the process and keep the processor idle. An optimal data size will support constant data reloads to keep the processor busy. A rough approximation of the instruction size, the time needed to move the data and a mechanism to detect an idling SPE is needed to optimize the program.

\subsection{Sequential design}
\label{sec:serial_design}
This section will explain in detail how the serial version of the software works.
At first the input parameters are checked if they can be transformed to correct invocation parameters that can be passed to the alignment algorithm. If this fails the program will inform the user how to set up the parameters correct and then exit.
The invoking parameters are:

\begin{enumerate}
   \item a string\\ which sets the path and filename where the output of the program shall be stored
   \item a positive integer\\ that is a reference point to determine the size of the generated image
   \item a positive natural number to base 10\\ that sets the first value for the primality test
\end{enumerate}

Now the second argument is transformed to the actual size the chosen algorithm can compute and corresponds to the nearest number that matches the algorithms scaling criteria which are defines as follows with $n$ as a positive number:
\begin{itemize}
   \item The Ulam spiral\\
      $(2/n)+1$
   \item The Hilbert curve\\
      $2^n$
   \item The N Curve\\
      $3^n$
\end{itemize}
For a recursive alignment algorithms of square shape, the scalability in general is $a^n$ where a is length of minimal iteration. This has to be changed if the alignment algorithms is not of square form.

The \emph{bmplib} library transforms the third argument into a \emph{mpz} (where \emph{mp} stands for Multiple Precision and \emph{z} represents the signed integer type) with base 10. This \emph{mpz} is needed to grant arbitrary large numbers to pass the primality check and may itself be of finite size.

In the next step, the \emph{gmplib} creates a bitmap (bmp) file of the size determined before. After the starting point for the algorithm is set, it is invoked with the number of iteration (the $n$ stated before) and the initial \emph{mpz}. The algorithm itself traverses the bmp and at every position checks if the corresponding number is a prime number or not. If so, the pixel is set to be black.
Now the generated bmp may be stored with the first given path and file name.
The thereby generated two dimensional pixel array contained within the bmp file is now transformed to other pixel arrays which form vectors through the bmp. The probability algorithm receives these pixel arrays and for every white pixel within one of these arrays the numerator is increased by one, where the denominator is increased with every pixel. After this transformation the array now is a fraction. All parallel vectors are now again set into an array they holds all probabilities for one of four vectors that stand at an angle of 45$^\circ$. Finally, they are now printed on the standard output stream and may be stored in a file.

The different alignment algorithms and the probability computation are executed one after another.

\subsection{Parallel design}
\label{sec:paralleldesign}
The decision that lead to this parallel design origin in the analysis of the X.264 codec parallelization. The report of this project can be found in \cite{BS08} and the analysis in \cite{self}. The design of the parallel version also relied on the profiling results of the sequential version as described in section \ref{sec:seq_ex}. From theses informations I decided to split the program at first along the functions and then along the data.

When splitting the functions, a first step is the swapping of the complete alignment algorithms with the primality test to the SPEs. This would leave the Power Processor Unit (PPU) idling and implies a high overhead of data generation. The potentially highly time intensive primality test then would be performed for every number at every SPE. In addition, the initial number would have to be transfered. This is indeed not a trivial task. An arbitrary large number needs arbitrary large memory and the data structure in which the \emph{gmplib} stores these numbers are neither trivial.
To avoid these two problems, the prime number generation is done at the PPU and the SPEs transfer the generated data and reorder it according to their algorithm.
The PPU provides the data in small portions and if there is additional data ready to transfer it will increase a counter that can be read by the SPE. With this method, the SPE only reads set data that will not change again. This avoids concurrency issues.

If the number of alignment algorithms is smaller than the number of SPEs (which is the case), the left over SPEs idle. In a more advanced sequential version the algorithms itself could be split into subparts. The treelike  recursion as seen in figure \ref{fig:hilbertalgorithm} can be parallelized quite efficiently. In this scenario, the algorithm would be split until a preset amount of iterations is left. These iteration are computed on different SPEs. The offset to the origin from which the algorithm starts has to be passed, so the subpart can produce the correct outcome. The final merging of these subparts can either be done by the calling SPE or by a simple bitwise \emph{or} operation at the PPU where a pixel is set, when any SPE states that the pixel is a prime. In practice this in any case should be only one SPE.

The spreading of single alignment algorithms to single SPEs also has the advantage that the code, which has to be present at the SPE is kept minimal. Every SPE only has the code it uses and not the complete set of algorithm from which has to be chosen.

In the next step - the computation of the probabilities - is parallelized with the method of data splitting. The computation is a small summation and can be done very fast and efficient.

There are two method how the vector can be assembled. Either the PPU prearranges the array or the SPE computes for itself, where to get the single values that form the vector.  The offset of the pixel data structure has to be studies carefully but can be done in parallel at the SPEs. This point is discussed in more detail at section \ref{sec:2darray}. The second approach is more complex to implement and I guess it would be the more efficient way to do. This decision of course has to be determined by profiling and testing.


Some points of the parallelization aspect have to be experienced live to value their impact on the software performance. Only experiences with the hard- and software can reveal bottlenecks and other problems with data alignment. These aspects are described in the section \ref{sec:par_imp}.

\subsection{Comparison}
\label{sec:designcomp}

To illustrate the differences and similarities between the sequential and the parallel design, these two charts which represents the control flow of the two programs are given:

\begin{figure}[H]
\begin{minipage}[t]{0.475\textwidth}
\centering
    \graphic{seqflow}{width=5cm}{Control flow of the sequential version.} 
\end{minipage}
\hfill
\begin{minipage}[t]{0.475\textwidth}
\centering
   \graphic{parflow}{width=5cm}{Control flow of the parallel version.}
\end{minipage}
\end{figure}

In both figures, decision in the form of \emph{if - then - else} statements are represented by a rhombus. An optional feature is indicated by a circle. The cylinder stands for data storage and the lying, shield like symbol is the standard output stream. The time in both figures proceed roughly from top to bottom where all cycles point to top, thus expressing the additional time spend for every iteration. Data structures used are \emph{BMP}  for the bitmap file, \emph{pixel **} for the two dimensional pixel array and \emph{INIT} for the initialization structure that holds all needed information to set up the program.

Figure \ref{fig:seqflow} shows one algorithm and one one probability vector computation. In anticipation to the parallel design, the program paths that resemble the shown one but with other data is illustrated by arrows without any destination to the right. This is the case for different alignment algorithms and for all vectors. With this in mind, the program can be seen as a tree where the algorithms are the first branch and the vectors as the following two branches.

Figure \ref{fig:parflow} is vertically split into the section PPU and SPEs. The PPU manages the data storage and performs the primality test. The SPEs do the alignment algorithms and the probability vector computation.

The chosen presentation as well as the de facto designs reveal no vast differences between the parallel and the sequential version. This of course was intentionally. Right from the start the parallel version was hold in mind and influenced some decision made for the serial version. The data parallelization is to complex and to repetitive to be viewed in an image that shall summarize the design. This holds, too for the function parallelization which also minimized in the visual representation. Both of these decision are made to be of minimal in reimplementation effort. The one thing that is visually represented analogous to the challenge it is, is the data distribution between the SPE and the PPU. The main memory representation moved from the left to the center where it is heavily accessed by the SPE and the PPU. One can also conclude from the image that the SPE is not kept busy all the time, even if the representation of time is quite variant in the figures.

%comparison

%\subsection{subsection}
%\subsubsection{subsubsection}
%\paragraph{paragraph}
%\subparagraph{subparagraph}
%List
%\begin{itemize}%enumerate
%	\item Punkt1
%	\item Punkt2
%\end{itemize}%enumerate

%\graphic{label}{caption}

%Ref to Picture:
%(see:\ref{fig:XXX})

%Table
%\begin{table}[H]
%\begin{longtable} {| l l  | }
%\hline
%Topic1		& Topic2 	\\ \hline
%Cell1		& Cell2		\\
%Cell1.1	& Cell2.1	\\
%\hline
%\end{longtable}
%\end{table}

\section{Implementation}
\label{sec:implementation}
Unfortunately it was not possible to create a proper parallel implementation of the project in the given time. I will, however, present the results of the serial version.
At first, the used libraries are introduced and their selection justified.
In the second part, the theoretical problem of the primality test to decide if a number is a prime or not is discussed. Afterwards the two implemented algorithms to produce a space filling curve are explained. The second algorithm which implements the Hilbert curve holds as an example of how to implement other space filling curves like the e curve or the n curve. Pattern detection is the penultimate point of this section. The decision not to use edge detection but hard math is outlined. The very last and largest point is about the difficulties and challenges of the implementation of the parallel version.

\subsection{Libraries}
\label{sec:libs}
This section highlights all libraries imported to create this project. Besides the standard library for exploiting the cell processor \emph{libspe} and \emph{mfcio} as introduced in \cite{cellguide}, two additional libraries were used. One to handle the bmp file which holds the generated data and one to handle large numbers.

\subsubsection{gmplib}
As stated in section \ref{sec:number_size}, the project requires the handling of arbitrary large numbers. Best practice here is "The GNU Multiple Precision Arithmetic Library" which handles arbitrary large and precise numbers. The library also provides several number theoretic functions - including the used primality test - that were used in the context of prime numbers. This point will be described in section \ref{sec:prim_test}.

\subsubsection{libbmp}
To create, manipulate and save bmp files, only few libraries exist. It is more common to find these functionality in a larger context merged with other image formats. This is because of the inefficiency of the bmp format. In the field it is almost extinct in favor to the \emph{png}, \emph{gif} or \emph{jpeg} format. Bmp does not use any compression and is straight forward with the method to store the image information. As set by the bmp standard as in \cite{bmp}, the bmp consists of a file header, an information header and the image information block. This block may include a color table and references to this table, but may as well be the pure color information for every pixel.

This simple procedure comes in handy to the needs of this project, a plain storage solution for data that is fast and easy to adapt and use. The images generated by the software are not expected to gain great advantages from compression. The choice fell on the "libbmp - BMP library" published under the GNU Lesser General Public License on \url{http://code.google.com/p/libbmp/}. The used Version 0.1.3 is unfinished and lacks several features but allows easy and fundamental changes without big effort. Without any restriction on function usage, it was possible to manipulate underlying data structures and extract parts of the data without concerning about nested or static functions allowing a fast integration and implementation changes.

This advantage had to be payed with some overhead at the bmp file size. In this project, the manipulated  information per pixel was binary. The only interesting thing was, if a pixel represents a prime number on not. This suggest a monochrome bmp, where one bit represents one pixel. This however was not featured by this library, but was neither by other libraries of similar size. The one bit information is now stored in one byte.

\subsection{Primality test}
\label{sec:prim_test}
As described in section \ref{sec:primality_test}, the primality test is highly inefficient if it is not probabilistic. The implementation of an own primality test that can handle the number representation of the \emph{gmplib} would have cost too much time. Luckily, the very same library already provided a primality test. The Miller-Rabin probabilistic primality test is applied several times to minimize the probability of false positives in the prime pattern. The theoretical possibility remains. However, the probability that several of these false positives are aligned as a pattern within the large test scope is neglectable and can be falsified by applying a non heuristic primality test, which is just a small change inside the code.

\subsection{Space filling curves}
\label{sec:curves}
The creation of the curves to which the primes were aligned to gave some challenges which will be examined in this section.
\subsubsection{Definition}
A space filling curve is defined as a path through every position in a discrete multi dimensional space with equal edge-length.%cube like

\medskip
To these space filling curves, several other criteria can be applied:
\begin{itemize}%enumerate
   \item Dimensions\\
      To how many dimension does the curve apply
   \item Continuity\\
      The path does only jump to adjacency squares
   \item Recursive\\
      The path can be generated by recursive application of a simple pattern
   \item Size restriction\\
      The curve can be scaled to any size, without losing any of its other criteria
\end{itemize}%enumerate

\subsubsection{Spiral algorithm}
The algorithm to align the prime numbers in a spiral around the center of the bitmap is fairly simple. At first, the center has to be determined and set. To actually have a center represented by a pixel, the bitmap height and width has to be odd. With every iteration of the algorithm one additional circumnavigation is done. By starting in one direction the first and second line - which lies rectangular to the first - have the length of one. The next two lines that continue the spiral in the same direction - either clockwise or counter clockwise - have one additional pixel. This pixel is the center that has not been passed by the first two lines. After these 4 line have been drawn, the length of the next line is also increased by one. In a two dimensional space the four rectangular direction have also been passed, thus the spiral is now facing in the very same direction as at the beginning. The algorithm now repeats until the calculated amount of iteration or wished pixed have been passed.

\subsubsection{Hilbert curve}
\graphic{hilbert_curve}{width=10.5cm}{Three Iteration of the Hilbert curve.}
Parallel to the criteria mentioned above, the Hilbert curve is 2 dimensional, continuous and recursive, but restricted to space sizes of $2^{n}$. The curves can be spaced to higher dimensions.

To understand the algorithm for the Hilbert curve, it is crucial to envision the derivations and changes between the single steps of the Hilbert curve. One iteration of the Hilbert curve passes a two by two pixel grid with the origin at the lower left pixel in this order: (0, 0), (1, 0), (1, 1), (0, 1). This is done by the transition of the given Axis (X, Y), starting at (0, 0) in this order: (+1, --), (-- , +1), (-1, --). These pixel scale transition are represented by a triangle facing in the direction to which the transition is done (\tt, \tl, \tb, \tr). So for traversing four pixels three transition were made.

This transition is represented by the symbol \sqr. The transition order may as well be done facing in other directions, and is then represented by the symbols \sqt, \sql and \sqb. Every one of these open squares have their own transition order. Their notation is as follows:

\begin{itemize}%enumerate
   \item \sqr $\Rightarrow$ \tr \tt \tl
   \item \sqt $\Rightarrow$ \tt \tr \tb
   \item \sql $\Rightarrow$ \tl \tb \tr
   \item \sqb $\Rightarrow$ \tb \tl \tt
\end{itemize}%enumerate

Unlike in other space filling curve, the direction of the transition is always the same. If the Hilbert curve is done with several iterations, the transition of the open square are interleaved with open squares as follows:

\begin{itemize}%enumerate
   \item \sqr $\Rightarrow$ \sqt  \tr  \sqr  \tt  \sqr  \tl \sqb
   \item \sqt $\Rightarrow$ \sqr  \tt  \sqt  \tr  \sqt  \tb  \sql
   \item \sql $\Rightarrow$ \sqb  \tl  \sql  \tb  \sql  \tr  \sqt
   \item \sqb $\Rightarrow$ \sql  \tb  \sqb  \tl  \sqb  \tt  \sqr
\end{itemize}%enumerate

Thus, the Hilbert curve is described as a grammar where the open squares are nonterminals and the triangles are terminals. However, the information of how often the nonterminal have to be exchanged has to be set globally and marks the number of iterations.

\graphic{hilbertalgorithm}{width=6cm}{Two Iteration of the Hilbert curve grammar in tree form.}

Figure \ref{fig:hilbertalgorithm} illustrates how the grammar can be interpreted as a tree with a depth equally to the iterations and a spreading factor equally to the minimum pixel size. If the tree is finished, inorder traversal reveals the sequence to rearrange the curve. 

%Thus a Hilbert curve of three iterations in generated as follows:
%\boxit{
%\sqt \hskip 10 pt  $\Rightarrow$ \\
%\sqr \tt \sqt \tr \sqt \tb \sql \hskip 10 pt $\Rightarrow$ \\
%\sqt \tr \sqr \tt \sqr \tl \sqb \hskip 10 pt \tt \hskip 10 pt \sqr \tt \sqt \tr \sqt \tb \sql \hskip 10 pt \tr \hskip 10 pt \sql \tb \sqb \tl \sqb \tt \sqr \hskip 10 pt \tb \hskip 10 pt \sqb \tl \sql \tb \sql \tr \sqt \hskip 10 pt $\Rightarrow$ \\
%\tt \tr \tb \tr \tr \tt \tl \tt \tr \tt \tl \tl \tb \tl \tt \hskip 10 pt \tt \\
%\tr \tt \tl \tt \tt \tr \tb \tr \tt \tr \tb \tb \tl \tb \tr \hskip 10 pt \tr \\
%\tl \tb \tr \tb \tb \tl \tt \tl \tb \tl \tt \tt \tr \tt \tl \hskip 10 pt \tb \\
%\tb \tl \tt \tl \tl \tb \tr \tb \tl \tb\tr \tr \tt \tr \tb 
%}{hilbert_grammar}{Derivation of three iterations of the Hilbert curve.}

Similar grammars and trees can be created for other space filling curves. If the curve can not be built by stepping from one field on the bitmap to an adjacent field, more advanced algorithms have to be used. They require a more complex computation between the steps to determine the step size. Jumping from one field to an adjacent, in respect to the current state of the pattern, is the serial approach. A parallel version would have to use a divide and conquer strategie as discussed in section \ref{sec:paralleldesign}.

\subsection{Pattern detection}
\label{sec:pattern}
The identification of visible lines within the Ulam spiral and other traversions of the prime number sequence, is set as a goal of this project. For further studies of the prime number sequence the identification of these lines can assist to evolve probabilistic primality tests which rely on the mathematical functions of the lines found in the Ulam spiral.

At first, it is of use to get an idea of what these visible lines mean in a mathematical way. The vector through an image can be described as a mathematical function or an algorithm that traverses the image. In the Ulam spiral these vectors lay at multiples of 45$^\circ$.
If one of these vectors can be identified by the eye, this means that there are significantly more points on this vector than on the others. It also may be case, that the closest parallel vector to the one seen has significantly less points and therefore the seen one gets a high contrast.

It is difficult to identify these lines with pattern detection algorithms from the field of image processing. A variety of edge detection algorithms like the Laplace filter or the Sobel operator can be found. These algorithms rely on continuous lines and blur the original image.$^{\cite{image}}$ This is not precise and reliable enough to enable further utilization. Therefore it is necessary to come up with numbers that describe the fact, that one can identify lines in the pattern. 

The implementation gives a probability to every vector. This statistic value approximates the probability, that a given point on this vector is prime number or not. The cumulation of several crossing vector then can produce an even more accurate probability, that a point represents a prime or not.

For small images and at the edges of an image, where a 45$^\circ$ vector just holds few values, this probability of course is inaccurate and has to be discarded. It was possible to clearly locate high probability vectors within the images. The corresponding figures can be found in section \ref{sec:results}.

\subsection{Parallel implementation}
\label{sec:par_imp}
The implementation of the parallel version of the software was very challenging and it turned out be be too complex to be finished in the time given. I will elaborate on these challenges and the way I addressed them. Some were implemented successfully, some had to stay in the design phase as reported in section \ref{sec:paralleldesign}. I ascribe my failure to a personal insufficiency of experience in programming the Cell processor in C and not to misconceptions in the design or defective assumptions about the software. To approve this, I will give a short word on my experiences in fore run to the project and then discuss three major challenges I faced.
I learned C because of this project and usually favor higher abstraction programming languages. Starting with Java, Lisp  and after some dissipations with Python and Prolog I currently have most experience with Haskell. Haskell is functional, pure and strictly typed. From there - without stating any superiority of any of these languages - I am not very skilled with implicit type conversions and pointer arithmetics. Both of them are heavily used when programing for the Cell processor and both are core parts of my following narrated problems. I was able to identify the problems and came up with solutions but it took too much time to implement all of them, which finally lead to an incomplete implementation.

\subsubsection{Pointers}
The explicit and implicit use of pointers is essential. Generally when writing code for the Cell processor and in special when dealing with copious data structure like bitmaps which hold pixel arrays and header information at different non subsequently heap spaces. Pointers are also used to pass and address data in the main storage which can not be accessed by the SPE. The SPE has to transfer the data to its own local store. The clear distinction between the data and pointers to this data or within this data is crucial. The arithmetics on this data also has to be figured out carefully. Especially because own data structures have to be aligned when they need to be passed between the SPE and the PPU or vice versa.This alignment in combination with multiple type definitions and the conversion between these type definition is hard to track. Without data changes of the very same bit vector on the machine the ways of accessing them vary widely. In addition, the data may be interlaced with zeros to meet the alignment criteria. I could not rely on the data size within the data structure to deduce the pointer to the data within the structure without extensive testing.

The visual representation also varied with different type definitions. It occurred that multiple data were represented as the same due to its accessing data type. This in particular is hard to identify if the data is represented as zero on the output but actually is not.

\subsubsection{Two dimensional arrays}
\label{sec:2darray}
Two dimensional array are often used in the program. To access a single pixel of the bitmap the two dimensional array is an obvious choice because the two dimensions resemble the two dimensions of the image. To conclude from this analogousness that an array is one block of data is faulty.

On the other hand, the two dimensional array serves as a great data structure to exploit the Cell processor architecture. The distinction between the first dimension, which is an array of pointers, and the second dimension, which is the actual data referenced by the first dimension, comes in handy to the polling mechanism. This mechanism is triggered by the SPE to get data from the main memory. The typically large data behind a two dimensional array can - and probably has to be - obtained in a couple of requests. The transfer of the first dimension of the array already holds the pointers needed to request portions of the data needed.

The trick here is to be aware of the different sizes the data has in the first and in the second dimension and that every data lies within the second dimension. Pointers can only navigate in one of the many arrays of the second dimension and conclusions from one second dimension array pointer to another are only possible if the difference between the references of the referencing pointers in the fist dimension are taken into account. I suggest not to conclude any pointer positions in two distinct second dimension array but to traverse every single array. 

\subsubsection{Data transfer size}
The two functions used to transfer data from and to the SPEs are \emph{mfc\_ get} and \emph{mfc\_put}. Both of them are restricted in the size. This restriction however is not the local memory limit and the data required to transfer may exceed this limitation. For both of these functions, wrappers were implemented (\emph{putback} and \emph{getfrom}) that can be requested to transfer arbitrary large data. They split the data in half and recursively call themselves on the bisect data until the data size is fallen below the limit.
% These aspects include data transfer and data alignment.
% 2d array
% pixel size

%\subsectioncd {subsection}
%\subsubsection{subsubsection}
%\paragraph{paragraph}
%\subparagraph{subparagraph}
%List
%\begin{itemize}%enumerate
%	\item Punkt1
%	\item Punkt2
%\end{itemize}%enumerate

%\graphic{label}{caption}

%Ref to Picture:
%(see:\ref{fig:XXX})

%Table
%\begin{table}[H]
%\begin{longtable} {| l l  | }
%\hline
%Topic1		& Topic2 	\\ \hline
%Cell1		& Cell2		\\
%Cell1.1	& Cell2.1	\\
%\hline
%\end{longtable}
%\end{table}

\section{Experimanetal Results}
\label{sec:results}

\subsection{Sequencial execution}

\subsection{Parallel execution}

\subsection{Comparison}

%\subsection{subsection}
%\subsubsection{subsubsection}
%\paragraph{paragraph}
%\subparagraph{subparagraph}
%List
%\begin{itemize}%enumerate
%	\item Punkt1
%	\item Punkt2
%\end{itemize}%enumerate

%\graphic{label}{caption}

%Ref to Picture:
%(see:\ref{fig:XXX})

%Table
%\begin{table}[H]
%\begin{longtable} {| l l  | }
%\hline
%Topic1		& Topic2 	\\ \hline
%Cell1		& Cell2		\\
%Cell1.1	& Cell2.1	\\
%\hline
%\end{longtable}
%\end{table}

\section{Program Usage}
\label{sec:usage}

%\subsection{subsection}
%\subsubsection{subsubsection}
%\paragraph{paragraph}
%\subparagraph{subparagraph}
%List
%\begin{itemize}%enumerate
%	\item Punkt1
%	\item Punkt2
%\end{itemize}%enumerate

%\graphic{label}{caption}

%Ref to Picture:
%(see:\ref{fig:XXX})

%Table
%\begin{table}[H]
%\begin{longtable} {| l l  | }
%\hline
%Topic1		& Topic2 	\\ \hline
%Cell1		& Cell2		\\
%Cell1.1	& Cell2.1	\\
%\hline
%\end{longtable}
%\end{table}

\section{Conclusion}
\label{sec:conclusion}

\subsection{Prime number alignement}

\subsection{}

%\subsection{subsection}
%\subsubsection{subsubsection}
%\paragraph{paragraph}
%\subparagraph{subparagraph}
%List
%\begin{itemize}%enumerate
%	\item Punkt1
%	\item Punkt2
%\end{itemize}%enumerate

%\graphic{label}{caption}

%Ref to Picture:
%(see:\ref{fig:XXX})

%Table
%\begin{table}[H]
%\begin{longtable} {| l l  | }
%\hline
%Topic1		& Topic2 	\\ \hline
%Cell1		& Cell2		\\
%Cell1.1	& Cell2.1	\\
%\hline
%\end{longtable}
%\end{table}

%\bibliographystyle{plain}
%epilogue

\newpage
\bibliographystyle{alpha}
\bibliography{bib}

   
\end{document}


%List
%\begin{itemize}%enumerate
%	\item Punkt1
%	\item Punkt2
%\end{itemize}%enumerate

%\graphic{label}{caption}

%Ref to Picture:
%(see:\ref{fig:XXX})

%Table
%\begin{table}[H]
%\begin{longtable} {| l l  | }
%\hline
%Topic1		& Topic2 	\\ \hline
%Cell1		& Cell2		\\
%Cell1.1	& Cell2.1	\\
%\hline
%\end{longtable}
%\end{table}
