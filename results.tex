\section{Experimental Results}
\label{sec:results}
The results of the project and the topic it had are presented here. At first i will point to the outcome of the meta topic of prime number alignment. Afterward i will conclude the observations of the serial program which was used to design the parallel version. This is concluded in the last section and sadly has to turn out short in lack of a proper implementation.

\subsection{Prime number probability}
\label{sec:resprob}
As a byproduct of this project, some data about the probability of prime numbers on vectors in space filling curves was produced. The discoveries shall be just presented and not discussed in detail. This will be the task of others. At first i present a medium size image for every alignment algorithm implemented to give a visual idea. The images are produced with the software. The Hilbert and the N curve start at the upper left corner the Ulam spiral in the middle. The calling parameters were 250 for the size, which lead to the given actual sizes and 1 for the initial prime number. The image of the Ulam spiral can be seen in figure \ref{fig:ulam251}.

\begin{figure}[H]
\begin{minipage}[t]{0.475\textwidth}
\centering
    \graphic{ncurve243}{width=5cm}{Primes aligned on a 243 x 241 N curve.} 
\end{minipage}
\hfill
\begin{minipage}[t]{0.475\textwidth}
\centering
   \graphic{hilbert128}{width=5cm}{Primes aligned to a 128 x 128 Hilbert curve.}
\end{minipage}
\end{figure}

As seen on these images, we expect three different outcomes for the probabilistic interpretation of these images. The order of the probabilities is irrelevant. We do not seek for dependencies between to flanking vectors. Therefore a nominal scale is chosen. As stated before, the values at the very left and very right are not relevant and therefore the horizontal an vertical vectors are centered. Their size is constant and they do not have small nominators. Even if the exact point and values can not be retrieved by this representation, for the short interpretation one can get a good idea about the situation.
In all four images a clear distinction between the skew and the rectangular vector can be made. The rectangular vectors have a lower average probability and a smaller range. This might be a systematic appearance or an accordance of the three alignment algorithm which can not be assessed here.

\graphic{ulamprobs}{scale=1}{Probabilities within the Ulam spiral.}
The Ulam spiral holds no new information. It is used as a reference to the other two images. The lines one can see in the image are the high striking values above 20\%. Besides these values the distribution is quite homogeneous with a large range rang of appearing percentage values. In larger pictures some lines appear to be horizontal an vertical. These lines are more likely to appear in the center. This can also be confirmed in the probabilistic analysis. There is a peak at the center of the horizontal and vertical vectors.

\graphic{hilbertprobs}{scale=1}{Probabilities within the the Hilbert curve.}
The image of the Hilbert reveals no prominent lines. The probability analysis also is quite homogeneous but the range of found percentage values is more dense. No peeks in any direction here simply means no lines in the image.

\graphic{ncurveprobs}{scale=1}{Probabilities within the the N curve.}
The probabilistic distribution of the N curve is quite similar to the one of the Hilbert Curve. its density is a little lower, but there are no extreme values in the middle. The N curve image clearly reveals vertical lines which in fact are "empty lines" where no prime number can be found. I suppose this is because of the very nature of the N curve which traverses the space not from a corner but in horizontal lines. The distance between to points does not correspond with a small number difference as it is case with the other two space filling curves.
These empty lines are not very obvious in the distribution graph. They are in fact excluded for surveyance reasons. On all graphs there are values of zero and they are in fact too much to plot in a meaningful way. The only way to recognize this is the comparison of the plotted point in the Ulam spiral and the N curve. The Ulam spiral has an actual size of 251 x 251 px and the N curve is 243 x 243 px. Therefore both probabilistic distribution should have the same amount of plotted spots. However the Ulam spiral seems to have a lot more. The difference of plotted spots is roughly the amount of empty vectors in the N curve.

Other observation like the higher probabilities on the right side and the higher prime number density at the curve origin can be explained by the very nature of the curves or prime numbers but have to be omit here.

\graphic{alltime}{width=9cm}{Performance of the serial program.}
The performance scaling graph of the program was made before the design decisions of the parallel version. It is quite common graph that indicates slightly growing additional computation overhead in relation to the growth of input. This will change drastically if the primality test is replaced with a non probabilistic version.

\subsection{Sequential execution}
\label{sec:seq_ex}
After the implementation of the sequential version of the software a representative execution was profiled to determine a most effective parallelization strategy. The general parallelization Design as described in section \ref{sec:paralleldesign} was refined by these results to optimize the speedup in the parallel version.
The profiling was done with \emph{gprof}, which tracks the time spent in a function and the number of calls of every function. More information can be found at \cite{gprof}.
The output of this profiling is as follows:

\boxit{
\ttfamily

Flat profile:

Each sample counts as 0.01 seconds.
\begin{longtable} {r r r r r r l}
  \%  & cumulative&    self&         &     self&     total& \\%           
 time &  seconds &  seconds&    calls&   s/call&   s/call&  name    \\%
 21.49&      1.76&     1.76&        3&     0.59&     0.59&  bmp\_malloc\\%\_pixels
 13.92&      2.90&     1.14&        3&     0.38&     0.38&  prob45deg\\%
 13.68&      4.02&     1.12&        3&     0.37&     0.37&  prob0deg\\%
 12.94&      5.08&     1.06&        3&     0.35&     0.35&  prob135deg\\%
  9.22&      5.83&     0.76& 52261913&     0.00&     0.00&  bmp\_set\_pix\\%el
  6.84&      6.39&     0.56&        3&     0.19&     0.19&  prob90deg\\%
  5.37&      6.83&     0.44&        1&     0.44&     0.67&  spiral\\%
  5.31&      7.27&     0.43&        1&     0.43&     0.58&  hilbert\\%
  4.52&      7.64&     0.37& 49133184&     0.00&     0.00&  markprime\\%
  3.79&      7.95&     0.31&         &         &         &  main\\%
  1.71&      8.09&     0.14& 49144720&     0.00&     0.00&  bmp\_get\_hei\\%ght
  1.04&      8.18&     0.09&        1&     0.09&     0.13&  ncurve\\%
  0.18&      8.19&     0.01&         &         &         &  \_is\_big\_end\\%ian
  0.00&      8.19&     0.00&    11537&     0.00&     0.00&  bmp\_get\_wid\\%th
  0.00&      8.19&     0.00&        6&     0.00&     0.00&  uint32\_pow\\%
  0.00&      8.19&     0.00&        3&     0.00&     0.59&  bmp\_create\\%
  0.00&      8.19&     0.00&        3&     0.00&     0.00&  bmp\_create\\%\_standard\_color\_table
  0.00&      8.19&     0.00&        3&     0.00&     0.00&  bmp\_malloc\\%\_colors
  0.00&      8.19&     0.00&        3&     0.00&     1.29&  probs\\%
  0.00&      8.19&     0.00&        1&     0.00&     0.00&  bmp\_destroy\\%
  0.00&      8.19&     0.00&        1&     0.00&     0.00&  bmp\_free\_co\\%lors
  0.00&      8.19&     0.00&        1&     0.00&     0.00&  bmp\_free\_pi\\%xels
%CALL: 250, 600, 2100, 5250 - all algorithms, no file safe
\end{longtable}}{fig:gprofoutput}{Gprof output for the serial version.}

Noteworthy information from this is that the two of the three alignment algorithms \emph{spiral} and \emph{hilbert} take approximate the same time where \emph{ncurve} is faster. Also the four vector probability functions \emph{prob0deg} - \emph{prob135deg} take some time where \emph{prob90deg} is faster than the others. The time spend in memory allocation \emph{malloc} is reasonable but may be optimized in the parallel version. This is taken in account especially with the two dimensional arrays as described in \ref{sec:paralleldesign}.
Less expected were the functions \emph{bmp\_get\_hei[ght]} and \emph{bmp\_set\_pix[el]}. A look into the \emph{libbmp} revealed that these function were dependently to \emph{markprime} which is called to set prime pixels white. Whenever a pixel is set the height and width of the bmp is retrieved to navigate to the pixels position. This could be optimized by rewriting the \emph{libbmp} but was not discarded in respect to the projects scope.

\subsection{Parallel execution}
\label{sec:par_ex}
Since the parallel version does not work properly, there are no results besides the design analysis presented in section \ref{sec:paralleldesign} and the implementation challenges described in section \ref{sec:par_imp}.

\subsection{Comparison}
Since the parallel version does not work properly, a comparison of the serial and parallel version as intended is not possible. The comparison of the designs is already done in section \ref{sec:designcomp}.

%pixel size

%\subsection{subsection}
%\subsubsection{subsubsection}
%\paragraph{paragraph}
%\subparagraph{subparagraph}
%List
%\begin{itemize}%enumerate
%	\item Punkt1
%	\item Punkt2
%\end{itemize}%enumerate

%\graphic{label}{caption}

%Ref to Picture:
%(see:\ref{fig:XXX})

%Table
%\begin{table}[H]
%\begin{longtable} {| l l  | }
%\hline
%Topic1		& Topic2 	\\ \hline
%Cell1		& Cell2		\\
%Cell1.1	& Cell2.1	\\
%\hline
%\end{longtable}
%\end{table}
