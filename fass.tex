\subsection{Space filling curves}
\label{sec:curves}
\par{Definition}
A space filling curve is defined as a path through every position in a discrete multi dimensional space with equal edge-length.%cube like

To these space filling curves, several other criteria can be applied:
\begin{itemize}%enumerate
   \item Dimensions\\
      To how many dimension does the curve apply
   \item Continuity\\
      The path does only jump to adjacency squares
   \item Recursive\\
      The path can be generated by recursive application of a simple pattern
   \item Size restriction\\
      The curve can be scaled to any size, without losing any of its other criteria
\end{itemize}%enumerate

\subsubsection{Examples}
\label{subsec:curve_examples}

\graphic{peano_curve}{8}{Two Iteration of the Peano curve.}
The Peano curve is a 2 dimensional continuous and recursive space filling curve, restricted to space sizes of $3^{n}$.

The Hilbert and the Moore curve are 2 dimensional, continuous and recursive, but restricted to space sizes of $2^{n}$. They differ in their recursion. All three curves can be spaced to higher dimensions.
 
\graphic{hilbert_curve}{10.5}{Three Iteration of the Hilbert curve.}
\graphic{moore_curve}{10.5}{Three Iteration of the Moore curve.}
The E curve is a bit more complex. It is restricted to spaces of $5^{n}$ and is continuous in the "Moore neighborhood" that holds all points with at least an adjacent corner, not only an adjacient side, a sin the "Von Neumann neighborhood".
\graphic{e_curve}{10.5}{Second iteration of the E curve.}

The z curve is a non continuous, 2 dimensional recursive curve that can be applied to multidimensional spaced of size $2^{n}$.
\graphic{z_curve}{10.5}{Three Iteration of the Z curve.}

Curves that fulfill all above mentioned criteria are quite simple. They either traverse one dimension after another or circle around the center of a subset of all dimensions and then iterate in the next dimension. In the second dimension this results either in a snake like passing from top to bottom or in a spiral form, as seen in \ref{fig:number_spiral}.

In other definition of the space filling curve, which are not restricted to an equal edge-length space, curves like the \emph{Dragon curve} or the \emph{Sierpi\`nski curve} can be mentioned.


%\subsection{subsection}
%\subsubsection{subsubsection}
%\paragraph{paragraph}
%\subparagraph{subparagraph}