\section{Design}
\label{sec:design}
The design of the software was aligned to the given requirements and will be presented in the same order. In a conflating step, the given design deliberations are merged into an overall design.

\subsection{Task analysis}
The additional requirements as written down in section \ref{sec:additional_req} lead to an initial phase where all given tasks were examined theoretically. The considerations of this phase are concluded in this section.

\subsubsection{Primality tests}
\label{sec:primality_test}
A more theoretical problem is the primality test. It tests if a given number \emph{n} is a prime number, but does not lead to its prime factors and therefor is a bit faster than integer factorization. This test, applied to a list of all integers, leads to a list of all primes and is therefore suitable to create an Ulam spiral. Probabilistic test hold a chance for false positives, for their prooof is restricted by a probability. An error within this probability may lead to unintended patterns in the Ulam spiral. This problem is considered in section \ref{sec:prim_test} where the implementation of the primality test is discussed.

Before the decision about which primality test to use, some of these test were examined:

\paragraph{Naive primality test:}
The simplest and less effective primality test for the number $n$ is to check all numbers up to \emph{n}, if \emph{n} is divisible by any of these numbers. A small improvement would be to  test just numbers $< \sqrt{n}$, because at least one of the factors of \emph{n} hast to be in this range.

%A famous deterministic primality test is the "Adleman-Pomerance-Rumely primality test". It was later improved by Henri Cohen and Arjen Lenstra and can test primality of an integer n in time:
    $(\log n)^{O(\log\,\log \,\log n)}$. 

%\paragraph{The AKS primality test}, also known as Agrawal-Kayal-Saxena primality test ord cyclotomic AKS test is a deterministic primality-proving algorithm and was published in \cite{primeisp}. And is descibed as follows:

%\begin{quote}
%Input:\\
%\noindent\hspace*{12mm} integer n $>$ 1.\\
%1. If (n = ab for a $\in$ N and b $>$ 1), output COMPOSITE.\\
%2. Find the smallest r such that or (n) $>$ log2 n.\\
%3. If 1 $<$ (a, n) $<$ n for some a $\leq$ r, output COMPOSITE.\\
%4. If n $\leq$ r, output PRIME.1\\
%5. For a = 1 to $\phi$(r) log n do\\
%   \noindent\hspace*{12mm} if ((X + a)n = X n + a (mod X r - 1, n)),\\
%   \noindent\hspace*{24mm}output COMPOSITE;\\
%6. Output PRIME;
%\end{quote}

\paragraph{Sieve of Eratosthenes:}
An alternative way of generating all prime numbers is the so called "Sieve of Eratosthenes". It works efficiently for smaller primes.

From a list of all natural numbers $> 2$ the three following steps will reveal all prime numbers:

\begin{itemize}%enumerate
   \item The first number in the list is a prime
   \item Strike this number and all multiples of the current number from the list
   \item repeat  
\end{itemize}%enumerate

This test scales quite well for small numbers up to 10,000. The test of every number includes the testing of all numbers below which results in massive computation overhead if the number exceeds the algorithms scalability.

\paragraph{Miller-Rabin probabilistic primality test:}
The Miller-Rabin probabilistic primality test was first published in \cite{MRPT} and verified in \cite{MRVer}. The running time of this algorithm is $O(k$ $log^{3} n)$, where k is the number of different values tested. Therefore it is quite performant while have the chance of false positives.

\subsubsection{Number size}
\label{sec:number_size}
As proven by Euclid, there are infinitive many prime numbers. To verify possible patterns, the large prime numbers are of special interest. Only if a regularity also holds with large numbers, it is of interest to proof this regularity mathematically.
The prime number theorem (PNT) describes the asymptotic distribution of prime numbers. The PNT, based on the formula of prime distribution by Gauss, gives an approximation of the density of prime numbers. According to Chebyshev, this density seems do decrease with larger numbers, as there are larger and larger prime gaps. This may confirm regularities and also may be conformed by found regularities.

From the perspective of a programming language, large numbers include growing complexity in their handling. In theory, every number is only restricted by the memory size that holds the number. In practice, the limiting factor is the standard which defines the representation of the number and the size of the smallest cache the number passes. This normally is limited to 32bit on old processor architectures and 64 bit in modern architectures like the PS3.

For prime numbers, the problem only scales on the set of all positive natural numbers, which is a quite smaller problem than simulating continuous numbers and handling discretization errors. The  very common problem of handling large natural number is solved by various libraries, but has to be implemented with care. Buffer overflows on too small number ranges highly harm the result of the computation and have to be avoided.

\subsubsection{Testing}
\label{sec:tests}
To have a basis on which the pattern searches may be performed, several changes to the original Ulam spiral were taken in account:

\begin{itemize}%enumerate
   \item Starting number
   \item Size of the space
   \item Curve
   \item Number sequence
   \item Dimension of the curve   
\end{itemize}%enumerate

In rising complexity and time consumption, the last two point were dropped to future works, but the starting number, the size of the space and another two curves were implemented and tested.

\subsubsection{Storage}
\label{sec:concurrency}
The architecture of the PS3 with its Synergistic Processing Elements (SPE) limits the amount of data that can be processed. 256KB embedded SRAM have to store the instructions and data.$^{\cite{PTC}}$
Large data has to be broken down to chunks of this size at most. Context switches and data reloads will slow down the process and keep the processor idle. An optimal data size will support constant data reloads to keep the processor busy. A rough approximation of the instruction size, the time needed to move the data and a mechanism to detect an idling SPE is needed to optimize the program.

\subsection{Sequential design}
\label{sec:serial_design}
This section will explain in detail how the serial version of the software works.
At first, the input parameters are checked if they can be transformed to correct invocation parameters that can be passed to the alignment algorithm. If this fails the program will inform the user how to set up the parameters correctly and then exit.
The invoking parameters are:

\begin{enumerate}
   \item a string\\ which sets the path and filename where the output of the program shall be stored
   \item a positive integer\\ that is a reference point to determine the size of the generated image
   \item a positive natural number to base 10\\ that sets the first value for the primality test
\end{enumerate}

Now, the second argument is transformed to the actual size the chosen algorithm can compute and corresponds to the nearest number that matches the algorithms scaling criteria which are defined as follows with $n$ as a positive number:
\begin{itemize}
   \item $(2/n)+1$ for the Ulam spiral
   \item $2^n$ for the Hilbert curve
   \item $3^n$ for the N Curve
\end{itemize}
In general, the scalability is $a^n$ when the algorithm is recursive and of square shape. In this case, $a$ is the length of minimal iteration. This has to be changed if the alignment algorithm is not of square form.

The \emph{bmplib} library transforms the third argument into a \emph{mpz} (where \emph{mp} stands for Multiple Precision and \emph{z} represents the signed integer type) with base 10. This \emph{mpz} is needed to grant arbitrary large numbers to pass the primality check and may itself be of finite size.

In the next step, the \emph{gmplib} creates a bitmap (bmp) file of the size determined before. After the starting point for the algorithm is set, it is invoked with the number of iteration (the $n$ stated before) and the initial \emph{mpz}. The algorithm itself traverses the bmp and at every position checks if the corresponding number is a prime number or not. If so, the pixel is set to be black.
Now the generated bmp may be stored with the first given path and file name.
The thereby generated two dimensional pixel array contained within the bmp file is now transformed to other pixel arrays which form vectors through the bmp. The probability algorithm receives these pixel arrays and for every white pixel within one of these arrays the numerator is increased by one, where the denominator is increased with every pixel. After this transformation the array is now a fraction. All fractions of the parallel vectors are now again set into an array. For every one of four vector directions that stand at an angle of 45$^\circ$ one array is created. Finally, the fraction values are now printed on the standard output stream and may be stored in a file.

The different alignment algorithms and the probability computation are executed one after another.

\subsection{Parallel design}
\label{sec:paralleldesign}
The decision that lead to this parallel design originate in the analysis of the X.264 codec parallelization. The report of this project can be found in \cite{BS08} and the analysis in \cite{self}. The design of the parallel version also relied on the profiling results of the sequential version as described in section \ref{sec:seq_ex}. From theses informations I decided to split the program at first along the functions and then along the data.

When splitting the functions, a first step is the swapping of the complete alignment algorithms with the primality test to the SPEs. This would leave the Power Processor Unit (PPU) idling and implies a high overhead of data generation. The potentially highly time intensive primality test then would be performed for every number at every SPE. In addition, the initial number would have to be transfered. This is indeed not a trivial task. An arbitrary large number needs arbitrary large memory and the data structure in which the \emph{gmplib} stores these numbers are not trivial either.
To avoid these two problems, the prime number generation is done at the PPU and the SPEs transfer the generated data and reorder it according to their algorithm.
The PPU provides the data in small portions and if there is additional data ready to transfer, it will increase a counter that can be read by the SPE. With this method, the SPE only reads set data that will not change again. This avoids concurrency issues.

If the number of alignment algorithms is smaller than the number of SPEs (which is the case), the left over SPEs idle. In a more advanced sequential version, the algorithms itself could be split into subparts. The treelike  recursion as seen in figure \ref{fig:hilbertalgorithm} can be parallelized quite efficiently. In this scenario, the algorithm would be split until a preset amount of iterations is left. These iteration are computed on different SPEs. The offset to the origin from which the algorithm starts has to be passed, so the subpart can produce the correct outcome. The final merging of these subparts can either be done by the calling SPE or by a simple bitwise \emph{or} operation at the PPU where a pixel is set, when any SPE states that the pixel is a prime. In practice, this should be only one SPE in any case.

The spreading of single alignment algorithms to single SPEs also has the advantage that the code, which has to be present at the SPE is kept minimal. Every SPE only has the code it uses and not the complete set of algorithm from which has to be chosen.

In the next step - the computation of the probabilities - is parallelized with the method of data splitting. To get a most precise result, the probability is stored as a fraction. The nominator and denominator are sequencially increased which is a very quick and efficient computation.

There are two method how the vector can be assembled. Either the PPU prearranges the array or the SPE computes for itself where to get the single values that form the vector.  The offset of the pixel data structure has to be studied carefully but can be done in parallel at the SPEs. This point is discussed in more detail at section \ref{sec:2darray}. The second approach is more complex to implement and I suppose it would be the more efficient way to do. This decision of course has to be determined by profiling and testing.

Some points of the parallelization aspect have to be experienced live to value their impact on the software performance. Only experiences with the hard- and software can reveal bottlenecks and other problems with data alignment. These aspects are described in the section \ref{sec:par_imp}.

\subsection{Comparison}
\label{sec:designcomp}

To illustrate the differences and similarities between the sequential and the parallel design, these two charts which represents the control flow of the two programs are given:

\begin{figure}[H]
\begin{minipage}[t]{0.475\textwidth}
\centering
    \graphic{seqflow}{width=5cm}{Control flow of the sequential version.} 
\end{minipage}
\hfill
\begin{minipage}[t]{0.475\textwidth}
\centering
   \graphic{parflow}{width=5cm}{Control flow of the parallel version.}
\end{minipage}
\end{figure}

In both figures, decision in the form of \emph{if - then - else} statements are represented by a rhombus. An optional feature is indicated by a circle. The cylinder stands for data storage and the lying, shield like symbol is the standard output stream. The time in both figures proceed roughly from top to bottom where all cycles point to top, thus expressing the additional time spend for every iteration. Data structures used are \emph{BMP}  for the bitmap file, \emph{pixel **} for the two dimensional pixel array and \emph{INIT} for the initialization structure that holds all needed information to set up the program.

Figure \ref{fig:seqflow} shows one algorithm and one probability vector computation. In anticipation to the parallel design, the program paths that resemble the shown one but with other data are illustrated by arrows without any destination to the right. This is the case for different alignment algorithms and for all vectors. With this in mind, the program can be seen as a tree where the algorithms are the first branch and the vectors as the following two branches.

Figure \ref{fig:parflow} is split vertically into the sections PPU and SPEs. The PPU manages the data storage and performs the primality test. The SPEs do the alignment algorithms and the probability vector computation.

The chosen presentation as well as the de facto designs reveal no vast differences between the parallel and the sequential version. This of course was intended. Right from the start, the parallel version was held in mind and influenced some decisions made for the serial version. The data parallelization is too complex and too repetitive to be viewed in an image that summarizes the design. This also is true for the function parallelization which also minimized in the visual representation. Both of these decision are made to generate minimal reimplementation effort. The one thing that is visually represented analog to the challenge it is, is the data distribution between the SPE and the PPU. The main memory representation moved from the left to the center where it is heavily accessed by the SPE and the PPU. One can also conclude from the image that the SPE is not kept busy all the time, even if the representation of time is quite variant in the figures.

%comparison

%\subsection{subsection}
%\subsubsection{subsubsection}
%\paragraph{paragraph}
%\subparagraph{subparagraph}
%List
%\begin{itemize}%enumerate
%	\item Punkt1
%	\item Punkt2
%\end{itemize}%enumerate

%\graphic{label}{caption}

%Ref to Picture:
%(see:\ref{fig:XXX})

%Table
%\begin{table}[H]
%\begin{longtable} {| l l  | }
%\hline
%Topic1		& Topic2 	\\ \hline
%Cell1		& Cell2		\\
%Cell1.1	& Cell2.1	\\
%\hline
%\end{longtable}
%\end{table}
