\section{Introduction}
\label{sec:intro}

The Electronic Frontier Foundation awarded \$100,000 for the first discovered prime number with at least 10,000,000 digits and the Great Internet Mersenne Prime Search ("GIMPS") won this prize in October 2009. \$100,000 and \$250,000 are awarded to the discovery of the first prime numbers with 100,000,000 and 1,000,000,000 decimal digits.

$2^{43,112,609}-1$ is the largest currently known prime number. Its discovery consumed 12 Years of continuous computing time with 29 trillion calculations per second and lead to 12,978,189 digits in decimal notation and was discovered by the Great Internet Mersenne Prime Search ("GIMPS").
Computing with prime numbers in general isn't trivial. Their size easily exceeds normal integer limitations and the algorithms to find and prove them are hard.

The reason behind this monetary appeal is the unsolved question if prime numbers follow a specific pattern to which they appear. This is one of the great question of number theory and up to a few years ago it was so solely. But since the integer factorization is the core of every encryption, is a question of great interest for computer science.  

Euclid has proven that there are infinitely many prime numbers, but there is no function which creates all possible prime numbers. Maybe this function can not exist, maybe this question is not decidable.\cite{zahlentheorie} However, it is possible to tell where between two prime numbers a gap of a given size is. These prime gaps prompt other mathematical questions but may give a hint to a pattern.

Another big discovery towards a pattern of prime numbers is the Ulam Spiral. In 1963 Stanislav Ulam was bored and started marking prime numbers on a square spiral as in \ref{fig:ulam_spiral_numbers}. \ref{fig:ulam_400_400} show a 400 by 400 Ulam spiral. At angles of 45° lines evolved. Up to now, it is not clear why these lines appear, but many people started their own search and came up with variation of the Ulam Spiral. The starting number may vary, the process was applied iteratively\cite{web} and other ways of illustration were chosen\cite{web}.

\begin{figure}[H]
\begin{minipage}[t]{0.475\textwidth}
\centering
    \graphic{ulam_spiral_numbers}{width=5cm}{The Ulam Spiral center.} 
\end{minipage}
\hfill
\begin{minipage}[t]{0.475\textwidth}
\centering
   \graphic{ulam251}{width=5cm}{A 251 x 251 Ulam spiral.}
\end{minipage}
\end{figure}

The search for patterns, the primality test, the vast search space where these patterns may evolve and the necessity to verify once found regularities against various variables is a fine problem to solve in parallel.
 
%\subsection{subsection}
%\subsubsection{subsubsection}
%\paragraph{paragraph}
%\subparagraph{subparagraph}


%\subsection{subsection}
%\subsubsection{subsubsection}
%\paragraph{paragraph}
%\subparagraph{subparagraph}